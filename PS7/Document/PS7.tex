\documentclass{article} % kind of document 
\usepackage[utf8]{inputenc} %encoding of choice
\usepackage[american]{babel} %language of choice
\usepackage[p,osf]{cochineal}
\usepackage{fancyhdr} %for header
\usepackage{amsmath, tabu} %math mode
\usepackage{mathtools}
\usepackage{extarrows} % for more options with arrows
\usepackage{amssymb} %math symbols
\usepackage{dsfont} %specifically for the indicator function symbol
\usepackage{xcolor} %to color text
\usepackage{amsthm} %math theorem
\usepackage{tikz}
\usepackage{caption}
\usepackage{multirow}
\usepackage[bottom]{footmisc}
% \usepackage[dvipsnames]{xcolor}
\usepackage{enumerate} %make lists
\usepackage{graphicx} %insert images
\usepackage{float} %to fix image position
\usepackage{moreverb} %to make boxes
\usepackage{hyperref} %to create hyperlinks
\usepackage{lipsum} %lorem ipsum package
\usepackage{setspace} % to use singlespace below in the solution environment
\usepackage[shortlabels]{enumitem}
\usepackage{parskip}
\usepackage[us]{datetime} %package for setting due date in US format
\newdate{duedate}{19}{10}{2021} %to set a due date
\allowdisplaybreaks
\usepackage[margin=1in]{geometry}
\pagestyle{fancy}


\lhead{Due: \displaydate{duedate}}
\chead{ECON 899 -- Problem Set 7 (SMM)}
\rhead{Danny, Hiroaki, Mitch, Ryan, Yobin}


\DeclareMathOperator*{\E}{\mathbb{E}} %ease of writing e and E
\newcommand{\e}{\mathrm{e}}
\newcommand{\ct}{\mathsf{c}}
\newcommand{\Z}{\mathbb{Z}}
\newcommand{\R}{\mathbb{R}}
\newcommand{\N}{\mathbb{N}}
\newcommand{\ifn}{\mathds{1}}
\newcommand{\X}{\mathbf{X}}
\newcommand{\Y}{\mathbf{Y}}
\newcommand{\one}{\mathbf{1}}
\newcommand\numberthis{\addtocounter{equation}{1}\tag{\theequation}}
\newcommand*\widebar[1]{\overline{#1}} % to get a widebar
\theoremstyle{definition}
\newtheorem{theorem}{theorem} % Theorem display format
\newtheorem{problem}[theorem]{Exercise} % Problem display format, last bracket sets display choice

\newenvironment{solution}[1][Answer]{\begin{singlespace}\underline{\textbf{#1:}}\quad }{\ \rule{0.3em}{0.3em}\end{singlespace}} % Answer format

\newenvironment{solutions}[1][Proof]{\begin{singlespace}\underline{\textbf{#1:}}\quad }{\ \rule{0.3em}{0.3em}\end{singlespace}} % Answer format

\begin{document}
	
	\begin{enumerate}
		\item Derive the following asymptotic moments associated with $ m_3(x) $ : mean, variance, first order auto-correlation. Furthermore, compute $ \nabla_b g(b_0) $. Which moments are informative for estimating $ b $?
		\begin{solution}
			We obtain the asymptotic moments as follows:
			\begin{align*}
				\E[x_t] & = \E[\rho_0 x_{t-1} + \epsilon_t] = \rho_0 \E[x_{t-1}] + \E[\epsilon_t] = \rho_0 \E[ \rho_0 x_{t-2} + \epsilon_{t-1} ] =  \rho_0^t \E[x_0] = 0.\\
				\E[(x_t - \E[x_t])^2] & = \E[x_t^2] = \E[(\rho_0 x_{t-1} + \epsilon_t)^2 ] \\ &= \rho_0^2 \E[x_{t-1}^2] + 2 \rho_0 \E[x_{t-1} \epsilon_t] + \E[\epsilon_t^2] \\ & = \rho_0^2 \E[ (\rho_0 x_{t-2} + \epsilon_{t-1})^2 ] + \sigma_0^2 \\ & = \rho_0^{2T} \E[x_0^2] + \sigma_0^2 \sum_{i=0}^{T-1} (\rho_0^2)^{i}  \rightarrow \frac{\sigma_0^2}{1 - \rho_0^2} = \frac{4}{3}.\\
				\E[(x_t - \E[x_t]) (x_{t-1} - \E[x_{t-1}]) ]&  =  \E[x_t x_{t-1}]  \\ &= \rho_0  \E[x_{t-1}^2]  + \E[ \epsilon_t x_{t-1} ] \\ & =   \rho_0 \E[ (\rho_0 x_{t-2} + \epsilon_{t-1})^2 ] \\ & =  \rho_0^3 \E[x_{t-2}^2] + \rho_0 \E[\epsilon_{t-1}^2]  \\ & = \rho_0^{2T - 1} \E[x_0^2] + \rho_0  \sigma_0^2 \sum_{i=0}^{T-2} (\rho_0^2)^{i}    \rightarrow  \frac{\rho_0 \sigma_0^2}{1 - \rho_0^2} = \frac{2}{3}   \\
			\end{align*}
			Moreover, we have that $ \nabla_b =
				\renewcommand\arraystretch{1.6}
			\begin{pmatrix}
				\frac{\partial}{\partial \rho} \\
				 \frac{\partial}{\partial \sigma^2}
			\end{pmatrix} $. Evaulating the derivative of $ g(\cdot) $ at $ b = b_0 $ gives us
			\begin{align*}
				\nabla_b g(b_0) = 
				\renewcommand\arraystretch{2}
				\begin{bmatrix}
					0 & 0 \\ \frac{2 \rho_0 \sigma_0^2}{(1 - \rho_0)^2} & \frac{1}{1 - \rho_0^2} \\  \frac{\sigma_0^2( 1 + 2 \rho_0^2)}{(1 - \rho_0^2)^2}  & \frac{\rho_0}{1 - \rho_0^2}
				\end{bmatrix}
				\xlongequal{\text{true values}}
				\begin{bmatrix}
					0 & 0 \\  4 & \frac{4}{3} \\  \frac{8}{3} & \frac{2}{3}
				\end{bmatrix}
			\end{align*}
		
			Both the variance and the first order correlation are informative for estimating $ b $. This is because we can estimate the true parameters given the two moments as follows,
			\begin{align*}
				\rho_0 = \frac{ \E[(x_t - \E[x_t]) (x_{t-1} - \E[x_{t-1}]) ] }{ 	\E[(x_t - \E[x_t])^2]  } && \sigma^2 =  \E[(x_t - \E[x_t])^2] -  \frac{ \E[(x_t - \E[x_t]) (x_{t-1} - \E[x_{t-1}]) ] }{ 	\E[(x_t - \E[x_t])^2]  }
			\end{align*}
		\end{solution}
		\item Simulate a series of “true” data of length $ T = 200 $ using (1). We will use this to compute
		$ M_T (x) $.
		
		\item Set H = 10 and simulate H vectors of length T = 200 random variables $ e_t $ from $ N(0, 1) $. We will use this to compute $ M_{TH}(y(b)) $. Store these vectors. You will use the same vector of random variables throughout the entire exercise. Since this exercise requires you to estimate $ \sigma^2 $, you want to change the variance of $ e_t $ during the estimation. You can simply use $ \sigma_{e_t} $ when the variance is $ \sigma^2 $.
		
		\item 
		
		\item Next we estimating the $ l = 2 $ vector $ b $ for the just identified case where $ m_2 $ uses the variance and autocorrelation. Given what you found in part (1), do you now think there will be a problem? If not, hopefully the standard error of the estimate of $ b $ as well as the J test will tell us something. Let’s see. For this case, perform steps (a)-(d) above.
		
		\item Next, we will consider the overidentified case where m3 uses the mean, variance and autocorrelation. Let’s see. For this case, perform steps (a)-(d) above. Furthermore, bootstrap the the finite sample distribution of the estimators using the following algorithm:
		\begin{enumerate}[i.]
			\item Draw $ \epsilon_t $ and $ e_t^h $ from $ N(0,1) $ for $ t = 1,2,\hdots,T $ and $ h = 1,2,\hdots,H $. Compute $ (\hat{b}_{TH}^1), \hat{b}_{TH}^2 $ as described.
			
			\item Repeat (e) using another seed.
		\end{enumerate}
	
	\end{enumerate}
\end{document}