\documentclass[landscape]{article} % kind of document 
\usepackage[utf8]{inputenc} %encoding of choice
\usepackage[american]{babel} %language of choice
\usepackage[p,osf]{cochineal}
\usepackage{fancyhdr} %for header
\usepackage{amsmath, tabu} %math mode
\usepackage{mathtools}
\usepackage{extarrows} % for more options with arrows
\usepackage{amssymb} %math symbols
\usepackage{dsfont} %specifically for the indicator function symbol
\usepackage{xcolor} %to color text
\usepackage{amsthm} %math theorem
\usepackage{tikz}
\usepackage{caption}
\usepackage{multirow}
\usepackage[bottom]{footmisc}
% \usepackage[dvipsnames]{xcolor}
%\usepackage{pythontex}
\usepackage{enumerate} %make lists
\usepackage{graphicx} %insert images
\usepackage{float} %to fix image position
\usepackage{moreverb} %to make boxes
\usepackage{hyperref} %to create hyperlinks
\usepackage{lipsum} %lorem ipsum package
\usepackage{setspace} % to use singlespace below in the solution environment
\usepackage[shortlabels]{enumitem}
\usepackage{parskip}
\usepackage[us]{datetime} %package for setting due date in US format
\newdate{duedate}{8}{11}{2021} %to set a due date
\allowdisplaybreaks
\usepackage[margin=.5in, landscape]{geometry}
\usepackage{rotating}
\usepackage{pdflscape}
\usepackage[paper=portrait,pagesize]{typearea}
\usepackage{jlcode}
\pagestyle{fancy}


\fancypagestyle{mylandscape}{
	\fancyhf{} %Clears the header/footer
	\fancyfoot{% Footer
		\makebox[\textwidth][r]{% Right
			\rlap{\hspace{.75cm}% Push out of margin by \footskip
				\smash{% Remove vertical height
					\raisebox{4.87in}{% Raise vertically
						\rotatebox{90}{\thepage}}}}}}% Rotate counter-clockwise
	\renewcommand{\headrulewidth}{0pt}% No header rule
	\renewcommand{\footrulewidth}{0pt}% No footer rule
}

\lhead{Due: \displaydate{duedate}}
\chead{ECON 899 -- PS1b -- }
\rhead{Danny, Hiroaki, Mitchell, Ryan, Yobin}


\DeclareMathOperator*{\E}{\mathbb{E}} %ease of writing e and E
\newcommand{\e}{\mathrm{e}}
\newcommand{\ct}{\mathsf{c}}
\newcommand{\Z}{\mathbb{Z}}
\newcommand{\R}{\mathbb{R}}
\newcommand{\N}{\mathbb{N}}
\newcommand{\ifn}{\mathds{1}}
\newcommand{\X}{\mathbf{X}}
\newcommand{\Y}{\mathbf{Y}}
\newcommand{\one}{\mathbf{1}}
\newcommand\numberthis{\addtocounter{equation}{1}\tag{\theequation}}
\newcommand*\widebar[1]{\overline{#1}} % to get a widebar
\theoremstyle{definition}
\newtheorem{theorem}{theorem} % Theorem display format
\newtheorem{problem}[theorem]{Exercise} % Problem display format, last bracket sets display choice

\newenvironment{solution}[1][Answer]{\begin{singlespace}\underline{\textbf{#1:}}\quad }{\ \rule{0.3em}{0.3em}\end{singlespace}} % Answer format

\newenvironment{solutions}[1][Proof]{\begin{singlespace}\underline{\textbf{#1:}}\quad }{\ \rule{0.3em}{0.3em}\end{singlespace}} % Answer format
\title{Econ899 PS1b}
\usepackage{listings}
\begin{document}
%\maketitle
\begin{enumerate}
\item The routine is coded in Julia. 
	
  The obtained log likelihood is -6942.805.
  
  The transpose of the score is
\begin{equation}
	\left[
	\begin{array}{c}
		-2605.9082518892865 \\
		-556.3196848948379 \\
		-1156.8594262530135 \\
		-222.81767101773977 \\
		-933.039979318137 \\
		-1215.1317422401712 \\
		-2109.626213790837 \\
		-948.0740374410863 \\
		-5049.875617650256 \\
		-4534.790470404961 \\
		-19401.89853086738 \\
		-19164.659456830384 \\
		-918.8553971099844 \\
		-351.75306280921296 \\
		-466.6888493111424 \\
		-582.4690752990825 \\
		-546.4113143620349 \\
	\end{array}
	\right]
\end{equation}

The Hessian matrix is too large to display, but is computed in the routine.
%\begin{tiny}
%	\begin{equation}
%		\left[
%		\begin{array}{ccccccccccccccccc}
%			-3224.6323170931637 & -880.4282370552784 & -1428.3856949992175 & -387.59093459559625 & -1305.6998486565487 & -1546.7677033060334 & -2619.4312932302696 & -1210.687131917734 & -6304.558251320736 & -5761.319479774926 & -23783.235245210555 & -23599.15535903185 & -1404.9888749434842 & -664.3517224229178 & -681.8501844814074 & -674.378931018232 & -582.9543820609548 \\
%			-880.4282370552784 & -880.4282370552784 & -0.0 & -10.076361578625994 & -404.23413474447085 & -421.32757532393583 & -686.2680547959762 & -331.9792492782152 & -1720.7476397292985 & -1655.0792540266505 & -6607.98944401972 & -6563.049857033983 & -390.078075551086 & -163.7777403901563 & -189.33729171154934 & -211.55444016783713 & -170.06932225388383 \\
%			-1428.3856949992175 & -0.0 & -1428.3856949992175 & -165.22677033814392 & -560.3440097381873 & -675.9773850917877 & -1192.1387611743 & -544.7133610454712 & -2796.018302626824 & -2551.2364457412073 & -10512.663121491 & -10428.544669004126 & -586.8866207257839 & -283.71101966745806 & -308.8773471223645 & -299.4399743267746 & -266.60578147545135 \\
%			-387.59093459559625 & -10.076361578625994 & -165.22677033814392 & -715.6050127102415 & -185.71963213990404 & -187.92797689361157 & -325.00290786402917 & -152.3491802406599 & -783.9173373851107 & -693.9968697171158 & -2739.1526435853884 & -2721.006111911267 & 43.925072005479564 & -59.79165501806744 & -104.65063370644353 & -92.37025235618046 & -77.26652364456983 \\
%			-1305.6998486565487 & -404.23413474447085 & -560.3440097381873 & -185.71963213990404 & -1305.6998486565487 & -693.5230338117503 & -973.1307655184933 & -501.4095827490417 & -2556.3483560054883 & -2592.7215636551587 & -9553.531122014616 & -9586.876508861225 & -501.9502655654747 & -214.5036191664594 & -290.98566119739195 & -299.4399743267746 & -192.87630650989607 \\
%			-1546.7677033060334 & -421.32757532393583 & -675.9773850917877 & -187.92797689361157 & -693.5230338117503 & -806.4437736759276 & -1231.5217025680984 & -585.3073343841065 & -3024.729094588563 & -2841.37273829907 & -11435.210328153036 & -11359.468963659121 & -660.4823984167546 & -312.0958838600313 & -326.3227234850444 & -325.5107161791306 & -282.9933855448839 \\
%			-2619.4312932302696 & -686.2680547959762 & -1192.1387611743 & -325.0029078640292 & -973.1307655184933 & -1231.5217025680984 & -2224.6526766119855 & -992.4601736297762 & -5125.8677423309455 & -4620.750060496868 & -19218.643540892648 & -19050.914689401558 & -1228.319287200248 & -545.1144490865397 & -551.4315903293557 & -540.1342687082149 & -477.38753651362896 \\
%			-1210.687131917734 & -331.9792492782152 & -544.7133610454712 & -152.3491802406599 & -501.4095827490417 & -585.3073343841065 & -992.4601736297762 & -528.6280472572579 & -2369.86193732619 & -2169.2686124365187 & -8869.944364220459 & -8798.625844084045 & -557.4931367061669 & -248.12425975073356 & -257.31586762977224 & -253.33447598163264 & -220.35282418037767 \\
%			-6304.558251320736 & -1720.7476397292985 & -2796.018302626824 & -783.9173373851107 & -2556.3483560054883 & -3024.729094588563 & -5125.8677423309455 & -2369.86193732619 & -12464.410119775508 & -11264.880715070343 & -46482.68652924107 & -46118.425573643355 & -2718.946424796173 & -1297.8353713269262 & -1333.2255193103879 & -1318.8728481837618 & -1134.647466736517 \\
%			-5761.319479774926 & -1655.0792540266505 & -2551.2364457412073 & -693.9968697171158 & -2592.7215636551587 & -2841.37273829907 & -4620.750060496868 & -2169.2686124365187 & -11264.880715070343 & -10834.69380513725 & -42612.431742976645 & -42303.332230405766 & -2419.5064504694287 & -1169.8410027867333 & -1223.7126724948937 & -1213.8820758328204 & -1033.5889330504042 \\
%			-23783.235245210555 & -6607.98944401972 & -10512.663121491 & -2739.1526435853875 & -9553.531122014616 & -11435.210328153036 & -19218.643540892648 & -8869.944364220459 & -46482.68652924107 & -42612.431742976645 & -176722.1299155234 & -175070.8479230722 & -10062.11901277724 & -4908.226203809249 & -5017.540465533479 & -4970.0704817676 & -4271.636079694827 \\
%			-23599.15535903185 & -6563.049857033983 & -10428.544669004126 & -2721.006111911268 & -9586.876508861225 & -11359.46896365912 & -19050.914689401558 & -8798.625844084045 & -46118.425573643355 & -42303.332230405766 & -175070.8479230722 & -174101.19245179262 & -9978.674938965785 & -4850.992462827001 & -4978.921952575023 & -4944.782256604929 & -4247.0536916281 \\
%			-1404.9888749434842 & -390.078075551086 & -586.8866207257839 & 43.925072005479564 & -501.9502655654747 & -660.4823984167546 & -1228.319287200248 & -557.4931367061669 & -2718.946424796173 & -2419.5064504694287 & -10062.11901277724 & -9978.674938965785 & -1404.9888749434842 & -292.9517805298065 & -306.9112277899499 & -305.7315561905012 & -268.37528887462446 \\
%			-664.3517224229178 & -163.7777403901563 & -283.71101966745806 & -59.79165501806744 & -214.5036191664594 & -312.0958838600313 & -545.1144490865397 & -248.12425975073356 & -1297.8353713269262 & -1169.8410027867333 & -4908.226203809249 & -4850.992462827001 & -292.9517805298065 & -664.3517224229178 & -0.0 & -0.0 & -0.0 \\
%			-681.8501844814074 & -189.33729171154934 & -308.8773471223645 & -104.65063370644353 & -290.98566119739195 & -326.3227234850444 & -551.4315903293557 & -257.31586762977224 & -1333.2255193103879 & -1223.7126724948937 & -5017.540465533479 & -4978.921952575023 & -306.9112277899499 & -0.0 & -681.8501844814074 & -0.0 & -0.0 \\
%			-674.378931018232 & -211.55444016783713 & -299.4399743267746 & -92.37025235618046 & -299.4399743267746 & -325.5107161791306 & -540.1342687082149 & -253.33447598163264 & -1318.8728481837618 & -1213.8820758328204 & -4970.0704817676 & -4944.782256604929 & -305.7315561905012 & -0.0 & -0.0 & -674.378931018232 & -0.0 \\
%			-582.9543820609548 & -170.06932225388383 & -266.60578147545135 & -77.26652364456983 & -192.87630650989607 & -282.9933855448839 & -477.38753651362896 & -220.35282418037767 & -1134.647466736517 & -1033.5889330504042 & -4271.636079694827 & -4247.0536916281 & -268.37528887462446 & -0.0 & -0.0 & -0.0 & -582.9543820609548 \\
%		\end{array}
%		\right]
%	\end{equation}
%\end{tiny}

	\item We obtain similar results using each of the two approaches. The Euclidean norm of the difference between scores obtained by the two approaches was around $ 1.40$ while the norm between the two Hessians was $ 0.35 $.
	
	
	\item The results obtained from implementing the Newton algorithm are displayed in part 4. The associated code is attached in the appendix. The \textit{NewtonAlg} function within  "functions.jl" file details the algorithm.
	
	
	\item  The computation speed between the three methods is compared below
	\begin{table}[H]
		\centering
		\begin{tabular}{c c }
			\hline
			Newton Method & 838.489 ms\\
			Quasi-Newton (BFGS) & 4.703 s\\
			Quasi-Newton (Simplex) & 2.935 s\\\hline
		\end{tabular}
	\end{table}
	
		The estimates of $ \beta $ under each of these methods is tabulated below. We can observe that the Newton Method yields estimates closest to the true values, and does so fastest among the three methods.
		
		\begin{tabular}{p{10em }p{10em } p{10em}} 
			\hline
			 Newton& BFGS& Simplex \\
			 \begin{equation*}
			 	\left[
			 	\begin{array}{c}
			 		-1.000 \\
			 		1.530e-7 \\
			 		8.4296e-8 \\
			 		1.086e-7 \\
			 		3.010e-8 \\
			 		1.569e-7 \\
			 		-1.938e-8 \\
			 		4.086e-8 \\
			 		1.278e-7 \\
			 		5.152e-8 \\
			 		-5.411e-8 \\
			 		3.651e-8 \\
			 		1.4033e-7 \\
			 		1.988e-7 \\
			 		1.170e-7 \\
			 		4.438e-8 \\
			 		2.374e-8 \\
			 	\end{array}
			 	\right]
			 \end{equation*} & 
		 	\begin{equation*}
		 		\left[
		 		\begin{array}{c}
		 			-6.056 \\
		 			0.867 \\
		 			0.527 \\
		 			0.595 \\
		 			0.163 \\
		 			0.871 \\
		 			-0.052 \\
		 			0.215 \\
		 			1.007 \\
		 			0.335 \\
		 			-0.284 \\
		 			0.189 \\
		 			0.758 \\
		 			1.152 \\
		 			0.770 \\
		 			0.379 \\
		 			0.2406 \\
		 		\end{array}
		 		\right]
		 	\end{equation*} &
	 		\begin{equation*}
	 			\left[
	 			\begin{array}{c}
	 				-1.953\\
	 				0.686\\
	 				0.318\\
	 				0.411\\
	 				 0.008\\
	 				-0.591\\
	 				-0.074\\
	 				-0.379\\
	 				0.279\\
	 				0.523\\
	 				-0.481\\
	 				0.233\\
	 				0.680\\
	 				0.344\\
	 				0.064\\
	 				-0.407\\
	 				-0.486\\
	 			\end{array}
	 			\right]
	 		\end{equation*}
	 	  \\
	 	\hline
		\end{tabular}

%  \begin{lstlisting}
%
%    1×17 Matrix{Float64}:
%    -2605.91  -556.32  -1156.86  -222.818  …  -466.689  -582.469  -546.411
%  \end{lstlisting}
%
%  The hessian is
%  \begin{lstlisting}
%    17×17 Matrix{Float64}:
%    -3224.63    -880.428    -1428.39   …   -681.85    -674.379    -582.954
%    -880.428   -880.428       -0.0        -189.337   -211.554    -170.069
%    -1428.39      -0.0      -1428.39       -308.877   -299.44     -266.606
%    -387.591    -10.0764    -165.227      -104.651    -92.3703    -77.2665
%    -1305.7     -404.234     -560.344      -290.986   -299.44     -192.876
%    -1546.77    -421.328     -675.977  …   -326.323   -325.511    -282.993
%    -2619.43    -686.268    -1192.14       -551.432   -540.134    -477.388
%    -1210.69    -331.979     -544.713      -257.316   -253.334    -220.353
%    -6304.56   -1720.75     -2796.02      -1333.23   -1318.87    -1134.65
%    -5761.32   -1655.08     -2551.24      -1223.71   -1213.88    -1033.59
%    -23783.2    -6607.99    -10512.7    …  -5017.54   -4970.07    -4271.64
%    -23599.2    -6563.05    -10428.5       -4978.92   -4944.78    -4247.05
%    -1404.99    -390.078     -586.887      -306.911   -305.732    -268.375
%    -664.352   -163.778     -283.711        -0.0       -0.0        -0.0
%    -681.85    -189.337     -308.877      -681.85      -0.0        -0.0
%    -674.379   -211.554     -299.44   …     -0.0     -674.379      -0.0
%    -582.954   -170.069     -266.606        -0.0       -0.0      -582.954
%  \end{lstlisting}
%  \item The numerical first derivative is
%  \begin{lstlisting}
%    1×17 Matrix{Float64}:
%    -2605.91  -556.32  -1156.86  -222.818  …  -466.689  -582.469  -546.411
%  \end{lstlisting}

\end{enumerate}
\newpage
\section*{Appendix}
 	The first codefile named "runfile.jl" runs the code.
 	\jlinputlisting{runfile.jl}
	
 	The second codefile named "functions.jl" contains the relevant functions.
 	\jlinputlisting{functions.jl}
\end{document}