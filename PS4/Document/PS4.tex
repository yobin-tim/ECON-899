\documentclass{article} % kind of document
\usepackage[utf8]{inputenc} %encoding of choice
\usepackage[american]{babel} %language of choice
\usepackage[p,osf]{cochineal}
\usepackage{fancyhdr} %for header
\usepackage{amsmath, tabu} %math mode
\usepackage{mathtools}
\usepackage{amssymb} %math symbols
\usepackage{dsfont} %specifically for the indicator function symbol
\usepackage{xcolor} %to color text
\usepackage{amsthm} %math theorem
\usepackage{tikz}
\usepackage{caption}
\usepackage{multirow}
\usepackage[bottom]{footmisc}
\usepackage[colorlinks=true, citecolor=blue, linkcolor=blue, urlcolor=blue]{hyperref} %to create hyperlinks
% \usepackage[dvipsnames]{xcolor}
\usepackage{enumerate} %make lists
\usepackage{graphicx} %insert images
\usepackage{float} %to fix image position
\usepackage{moreverb} %to make boxes
\usepackage{lipsum} %lorem ipsum package
\usepackage{setspace} % to use singlespace below in the solution environment
\usepackage[shortlabels]{enumitem}
\usepackage{parskip}
\usepackage[us]{datetime} %package for setting due date in US format
\newdate{duedate}{29}{09}{2021} %to set a due date
% \usepackage{jlcode}
\allowdisplaybreaks
\usepackage[margin=1in]{geometry}
\pagestyle{fancy}

\lhead{Due: \displaydate{duedate}}
\chead{ECON 899 -- Problem Set 3 }
\rhead{Danny, Mitchell, Ryan, Yobin, and Hiroaki}
\title{ECON 899 -- Problem Set 3}
\author{Danny, Mitchell, Ryan, Yobin, and Hiroaki}
\date{\today}

\DeclareMathOperator*{\E}{\mathbb{E}} %ease of writing e and E
\newcommand{\e}{\mathrm{e}}
\newcommand{\ct}{\mathsf{c}}
\newcommand{\Z}{\mathbb{Z}}
\newcommand{\R}{\mathbb{R}}
\newcommand{\N}{\mathbb{N}}
\newcommand{\ifn}{\mathds{1}}
\newcommand{\X}{\mathbf{X}}
\newcommand{\Y}{\mathbf{Y}}
\newcommand{\one}{\mathbf{1}}
\newcommand\numberthis{\addtocounter{equation}{1}\tag{\theequation}}
\newcommand*\widebar[1]{\overline{#1}} % to get a widebar
\theoremstyle{definition}
\newtheorem{theorem}{theorem} % Theorem display format
\newtheorem{problem}[theorem]{Exercise} % Problem display format, last bracket sets display choice

\newenvironment{solution}[1][Answer]{\begin{singlespace}\underline{\textbf{#1:}}\quad }{\ \rule{0.3em}{0.3em}\end{singlespace}} % Answer format

\newenvironment{solutions}[1][Proof]{\begin{singlespace}\underline{\textbf{#1:}}\quad }{\ \rule{0.3em}{0.3em}\end{singlespace}} % Answer format

\begin{document}
	\maketitle
	\subsubsection*{Exercise 1}
	\begin{enumerate}
		\item  Use the parametrization from the previous problem set. We continue to assume that labor supply is endogenous. Solve for the stationary equilibrium with social security $ (\theta^{SS}_0 = 0.11) $ and without it $ (\theta^{SS}_N = 0) $ following the algorithm described in the lecture notes (\textit{Step 1: Calculating the stationary competitive equilibrium}). Denote the initial distribution of agents over age, $ j $, asset holdings, $ a $, and productivity levels, $ z $, by $  \Gamma_0^{SS}(z,a,j;\theta_0^{SS}) $. Denote the welfare of agents alive in the initial steady state by $ V_0^{SS}(z,aj;\theta_0^{SS})$.
		
		\item  Compute the transition path of the economy using the algorithm in \textit{Step 2: Solving for the transition path} in the lecture notes. Try N = 30 for the
		number of periods it approximately takes to get to the new steady state. Obtain and store the value function for the generations in the initial steady state, $ V_0(z,a,j; \theta_0^{SS}, \theta_N^{SS}) $. Plot the transition paths of interest rate, wage, capital and effective labor. Comment on the results you obtain.
		\begin{solution}
			
		\end{solution}
		
		
		\item  What fraction of the overall population would support the reform? Compute and plot the measure of consumption equivalent variation for each age, $ EV_j $ , using $$ EV_j = \sum_z \int_a EV(z,a,j)  \Gamma_0^{SS}(z,a,j; \theta_0^{SS})  da, $$ with \[  EV(z,a,j) = \left(  \frac{ V_0(z,a,j; \theta_0^{SS}, \theta_N^{SS}) }{  V_0^{SS}(z,a,j; \theta_0^{SS}) } \right)^{\frac{1}{\gamma(1 - \sigma)}} . \] Discuss the results.		
		\begin{solution}
			
		\end{solution}
	\end{enumerate}


	\subsubsection*{Exercise 2}
	\begin{enumerate}
		\item Instead of considering an unexpected elimination of the social security system, assume that in $ t = 0 $ the government credibly announces that it is going to abolish the public pension system starting from $ t = 21 $ onwards. Thus, all individuals retired keep their social security benefits, but future retirees anticipate that they will receive only part or no social security benefits. Repeat steps (1)-(3) of exercise 1 to study how agents readjust their plans and how political support changes for the anticipated reform in 21 years. You will have to increase the number of transition periods (try $ N = 50 $). Discuss your results.
		\begin{solution}
				
		\end{solution}
	\end{enumerate}
\end{document}