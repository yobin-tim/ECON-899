\documentclass[landscape]{article} % kind of document 
\usepackage[utf8]{inputenc} %encoding of choice
\usepackage[american]{babel} %language of choice
\usepackage[p,osf]{cochineal}
\usepackage{fancyhdr} %for header
\usepackage{amsmath, tabu} %math mode
\usepackage{mathtools}
\usepackage{extarrows} % for more options with arrows
\usepackage{amssymb} %math symbols
\usepackage{dsfont} %specifically for the indicator function symbol
\usepackage{xcolor} %to color text
\usepackage{amsthm} %math theorem
\usepackage{tikz}
\usepackage{caption}
\usepackage{multirow}
\usepackage[bottom]{footmisc}
% \usepackage[dvipsnames]{xcolor}
%\usepackage{pythontex}
\usepackage{enumerate} %make lists
\usepackage{graphicx} %insert images
\usepackage{float} %to fix image position
\usepackage{moreverb} %to make boxes
\usepackage{hyperref} %to create hyperlinks
\usepackage{lipsum} %lorem ipsum package
\usepackage{setspace} % to use singlespace below in the solution environment
\usepackage[shortlabels]{enumitem}
\usepackage{parskip}
\usepackage[us]{datetime} %package for setting due date in US format
\newdate{duedate}{22}{11}{2021} %to set a due date
\allowdisplaybreaks
\usepackage[margin=.5in, landscape]{geometry}
\usepackage{rotating}
\usepackage{pdflscape}
\usepackage[paper=portrait,pagesize]{typearea}
\usepackage{jlcode}
\pagestyle{fancy}


\fancypagestyle{mylandscape}{
	\fancyhf{} %Clears the header/footer
	\fancyfoot{% Footer
		\makebox[\textwidth][r]{% Right
			\rlap{\hspace{.75cm}% Push out of margin by \footskip
				\smash{% Remove vertical height
					\raisebox{4.87in}{% Raise vertically
						\rotatebox{90}{\thepage}}}}}}% Rotate counter-clockwise
	\renewcommand{\headrulewidth}{0pt}% No header rule
	\renewcommand{\footrulewidth}{0pt}% No footer rule
}

\lhead{Due: \displaydate{duedate}}
\chead{ECON 899 -- PS2b -- }
\rhead{Danny, Hiroaki, Mitchell, Ryan, Yobin}

\DeclareMathOperator*{\E}{\mathbb{E}} %ease of writing e and E
\newcommand{\e}{\mathrm{e}}
\newcommand{\ct}{\mathsf{c}}
\newcommand{\Z}{\mathbb{Z}}
\newcommand{\R}{\mathbb{R}}
\newcommand{\N}{\mathbb{N}}
\newcommand{\ifn}{\mathds{1}}
\newcommand{\X}{\mathbf{X}}
\newcommand{\Y}{\mathbf{Y}}
\newcommand{\one}{\mathbf{1}}
\newcommand\numberthis{\addtocounter{equation}{1}\tag{\theequation}}
\newcommand*\widebar[1]{\overline{#1}} % to get a widebar
\theoremstyle{definition}
\newtheorem{theorem}{theorem} % Theorem display format
\newtheorem{problem}[theorem]{Exercise} % Problem display format, last bracket sets display choice

\newenvironment{solution}[1][Answer]{\begin{singlespace}\underline{\textbf{#1:}}\quad }{\ \rule{0.3em}{0.3em}\end{singlespace}} % Answer format

\newenvironment{solutions}[1][Proof]{\begin{singlespace}\underline{\textbf{#1:}}\quad }{\ \rule{0.3em}{0.3em}\end{singlespace}} % Answer format
\title{Econ899 PS1b}
\usepackage{listings}

\begin{document}
%\maketitle
\begin{enumerate}
\item See the QuadLL2 function in the julia code attached below.
\item For $T_i = 1$, the probability can be written as follows
  \begin{equation}
    \label{eq:1}
    Pr\left(\eta_{i0} < \frac{-\alpha_0 - X_i\beta - Z_{it}\gamma}{\sigma_0}\right)
  \end{equation}
  For $T_i = 2$, the probability is
  \begin{equation}
    \label{eq:2}
    Pr\left(\epsilon_{i0} < \alpha_0 + X_i\beta + Z_{it}\gamma, \epsilon_{i1} < - \alpha_1 - X_i\beta - Z_{it}\gamma \right)
  \end{equation}
  The conditional probability can be written as follows
  \begin{align*}
    & Pr\left(\epsilon_{i1} < - \alpha_1 - X_i\beta - Z_{it}\gamma |\epsilon_{i0} < \alpha_0 + X_i\beta + Z_{it}\gamma \right) \\  
    & = Pr\left(\eta_{i0} + \rho \sigma_0 \eta_{i1} < - \alpha_1 - X_i\beta - Z_{it}\gamma |\eta_{i0} < \frac{\alpha_0 + X_i\beta + Z_{it}\gamma}{\sigma_0} \right)\\
    & = Pr\left(\eta_{i1} < \frac{- \alpha_1 - X_i\beta - Z_{it}\gamma-\eta^*_{i0}}{\rho \sigma_0} \middle|\eta_{i0} < \frac{\alpha_0 + X_i\beta + Z_{it}\gamma}{\sigma_0} \right)
  \end{align*}
  where $\eta^*$ is a random variable from  $\eta_{i0} < \frac{\alpha_0 + X_i\beta + Z_{it}\gamma}{\sigma_0}$. Then, the probability of $T_i= 2$ is 
  \begin{equation}
    \label{eq:3}
    Pr\left(\eta_{i1} < \frac{- \alpha_1 - X_i\beta - Z_{it}\gamma-\eta^*_{i0}}{\rho \sigma_0} \middle|\eta_{i0} < \frac{\alpha_0 + X_i\beta + Z_{it}\gamma}{\sigma_0} \right) Pr\left(\eta_{i0} < \frac{\alpha_0 + X_i\beta + Z_{it}\gamma}{\sigma_0} \right) 
  \end{equation}
  Similarly, we can define probability in $T_i = 3,4$. Based on these probabilities, we can define the log likelihood. The funciton is GHKLL2. 

\item See the function AcceptRejectLL()

\item We got the following likelihoods: 
  \begin{itemize}
  \item  Quadrature method:-40992
  \item  GHK method:-60572 (Changes somewhat each time the function is called)
  \item  Accept/Reject method:-75167 (Changes somewhat each time the function is called)
  \end{itemize}
  The calculated likelihood is different under each method, but the order (-1e5) is the same.  
\item We got the following result. 
  \begin{align*}
    \alpha_0 &= 3.12, \alpha_1 = 0.88, \alpha_2 = 2.29 \\
    score_0 &= 0.00, rate spread = -0.25, large loan = -0.79, medium loan = -0.39 \\ 
    i refinance &= -0.04, age r = -0.42, cltv = 0.34, dti = 0.67, cu = -0.43 \\
    first mort &= 0.59, i FHA = -0.05, open year 2 = -0.70,open year 3 = 0.07 \\ 
    open year 4 &= 0.12, open year 5 =0.02 \\
    score0  &= 0.30, score1 = -0.16, score2 = 0.35, \rho = 0.57 
  \end{align*}
\end{enumerate}

\newpage
\section*{Appendix}
 	The first codefile named "runfile.jl" runs the code.
 	\jlinputlisting{runfile.jl}
	
 	The second codefile named "functions.jl" contains the relevant functions.
 	\jlinputlisting{functions.jl}
\end{document}